\documentclass[12pt,a4paper]{article}
\usepackage{graphicx}
\usepackage{ctex}
\usepackage{indentfirst}
%\graphicspath{{chapter/}{figures/}}
\usepackage{CJK}
\usepackage{amsmath}%数学

%\usepackage[colorlinks,linkcolor=red]{hyperref}%超链接

\usepackage{fancyhdr}  %使用fancyhdr包自定义页眉页脚
%\pagestyle{empty}
\pagestyle{fancy}
%\pagestyle{plain}%没有页眉,页脚放页数

\usepackage{titlesec}%设置章节标题与正文间距为2行
\titlespacing{\section}{0pt}{0pt}{2em}

\usepackage{enumerate}%项目编号

\renewcommand{\figurename}{图}%将figure改为图

\usepackage[]{caption2}%去掉图片编号后的":"
\renewcommand{\captionlabeldelim}{}

\renewcommand {\thefigure} {\thesection{}.\arabic{figure}}%图片索引该为按照章节

\renewcommand{\headrulewidth}{0.5pt}
\renewcommand{\footrulewidth}{0.4pt}
\lhead{}
\chead{}
\rhead{}
\lfoot{}
\cfoot{\thepage}
\rfoot{}

\usepackage{booktabs}%表格用

\usepackage{titlesec}%修改标题格式宏包
\titleformat{\section}{\centering\zihao{3}\bfseries}{\arabic{section}.}{0.5em}{}%修改section标题格式


\usepackage{multirow}%跨行表格
\usepackage{abstract}%摘要
\usepackage{setspace}   %行间距的宏包


\usepackage{makecell}%表格竖线连续

\def\I{\vrule width1.2pt}
%!\I 就可以代替| 来画表格了

%可固定下划线长度
\makeatletter
\newcommand\dlmu[2][4cm]{\hskip1pt\underline{\hb@xt@ #1{\hss#2\hss}}\hskip3pt}
\makeatother

\usepackage{float}%可以用于禁止浮动体浮动



%目录超链接
\usepackage[colorlinks,linkcolor=black,anchorcolor=blue,citecolor=black]{hyperref}

\usepackage{listings}%可以插入代码
\usepackage{xcolor}%语法高亮支持

%代码格式
\definecolor{dkgreen}{rgb}{0,0.6,0}
\definecolor{gray}{rgb}{0.5,0.5,0.5}
\definecolor{mauve}{rgb}{0.58,0,0.82}
\usepackage{fontspec}
\setmonofont{Consolas}
\lstset{ %
	numbers=left, 
	basicstyle=\tiny\ttfamily, 
	numberstyle=\tiny, 
	tabsize=4,
	numbersep=5pt, 
	keywordstyle= \color{blue!70}, %关键词为蓝色
	commentstyle=\color{gray}, %注释为灰色
	frame=shadowbox, % 框架阴影效果
	rulesepcolor= \color{ red!20!green!20!blue!20} ,
	escapeinside={\%*}{*)},
	xleftmargin=2em, % 边界选项
	xrightmargin=2em, % 边界选项
	aboveskip=1em, % 边界选项
	framexleftmargin=2em, % 边界选项
	breaklines,%过长代码自动换行
}




%设置页面格式
\usepackage[left=3.0cm, right=2.6cm, top=2.54cm, bottom=2.54cm]{geometry}
\begin{document}
	

%%%%%%%%%%%%%%%%%%%%%%%%%%%%%%
%% 封面部分
%%%%%%%%%%%%%%%%%%%%%%%%%%%%%%
\begin{titlepage}
%	\begin{minipage}[c]{0.75\textwidth}
%		\includegraphics[width=0.15\textwidth]{pic//logo//logo.jpg}
%%		{\LARGE 机械与电气工程学院}
%	\end{minipage}


\begin{figure}[H]
	\centering
	\includegraphics[scale=0.5]{pic//logo//logo.jpg} %1.png是图片文件的相对路径
%	\caption{DSP上位机程序} %caption是图片的标题
	\label{DSPswj} %此处的label相当于一个图片的专属标志,目的是方便上下文的引用
\end{figure}
\vspace{0.2cm}	
\centering

{\zihao{1}\songti{本科毕业论文(设计)}}

\vspace{2.5cm}

\begin{flushleft}
    {{\songti \zihao{-3} \qquad\qquad\qquad 课题名称}\quad{\zihao{-4}\dlmu[7.5cm]{*************}}\par}
	\vspace{0.5cm}
	{{\songti\zihao{-3} \qquad\qquad\qquad 学\qquad 院}\quad\dlmu[7.5cm]{机械与电气工程学院}\par}
	\vspace{0.5cm}
	{{\songti\zihao{-3} \qquad\qquad\qquad 专\qquad 业}\quad\dlmu[7.5cm]{*************}\par}
	\vspace{0.5cm}
	{{\songti\zihao{-3} \qquad\qquad\qquad 班级名称}\quad\dlmu[7.5cm]{************* }\par}
	\vspace{0.5cm}
	{{\songti\zihao{-3} \qquad\qquad\qquad 学生姓名}\quad\dlmu[7.5cm]{*************}\par}
	\vspace{0.5cm}
	{{\songti\zihao{-3} \qquad\qquad\qquad 学\qquad 号}\quad\dlmu[7.5cm]{*************}\par}
	\vspace{0.5cm}
	{{\songti\zihao{-3} \qquad\qquad\qquad 指导老师}\quad\dlmu[7.5cm]{*************}\par}
	\vspace{0.5cm}
	{{\songti\zihao{-3} \qquad\qquad\qquad 完成日期}\quad\dlmu[7.5cm]{*************}\par}
\end{flushleft}

\vspace{4cm}

{\songti \zihao{3} 教务处制}

\end{titlepage}

\renewcommand{\abstractname}{\scriptsize}
%%%%%%%%%%%%%%%%%%%%%%%%%%%%%%
%% 摘要和关键词部分
%%%%%%%%%%%%%%%%%%%%%%%%%%%%%%

\begin{center}
	{\zihao{3}\textbf{*************系统设计}}\par
	{\zihao{-4}\songti 电子信息工程 \quad 专业 \quad 电信151班 \quad ************* \par 
		指导教师:****}
\end{center}


\begin{onecolabstract}
	\noindent{}{\zihao{4}\textbf{摘要\qquad}}{\songti \zihao{-4}中文摘要内容********
		************************************************************
		***************************************************************************
	*************************************************************
	********************************************************
*****************************************************************}\par

\vspace{1ex}
	
	\noindent{}{\zihao{4}\textbf{关键词\qquad}}{\zihao{-4}\songti 关键词1;关键词2}\par
\end{onecolabstract}


\begin{onecolabstract}
%	\setlength\parskip{0em}
	\noindent{}{\zihao{4} \textbf{ABSTRACT\qquad}}{\zihao{-4}abstract in English *************
	*******************************************************************************************
*******************************************************************************************
*******************************************************************************************}\par
	
	
\vspace{1ex}
	
	\noindent{}{\zihao{4}\textbf{KEY WORDS\qquad}}{\zihao{-4}keywords1; keywords2}\par
\end{onecolabstract}


\newpage

%%%%%%%%%%%%%%%%%%%%%%%%%%%%%%
%% 目录部分
%%%%%%%%%%%%%%%%%%%%%%%%%%%%%%
\renewcommand{\contentsname}{\centerline{\zihao{-2}\textbf{目录}}}

\tableofcontents
\newpage

%%%%%%%%%%%%%%%%%%%%%%%%%%%%%%
%% 正文部分
%%%%%%%%%%%%%%%%%%%%%%%%%%%%%%
{
\setlength{\baselineskip}{23pt}


\section*{引\quad 言}
\addcontentsline{toc}{section}{1\quad  引言\tiny{\quad.\quad.\quad.\quad.\quad.\quad.\quad.\quad.\quad.\quad.\quad.\quad.\quad.\quad.\quad.\quad.\quad.\quad.\quad.\quad.\quad.\quad.\quad.\quad.\quad.\quad.\quad.\quad.\quad.\quad.\quad.\quad.\quad.\quad.\quad.\quad.\quad.\quad.\quad.\quad.\quad.\quad.\quad.\quad.\quad}}

引言部分


\newpage
\setcounter{section}{1}
\section{绪论}


\subsection{课题背景}
********************************************
\subsection{国内外研究进展情况}
****************************************************

\subsection{本课题的目的及意义}
*************

\subsection{总体研究思路}

******************************************************************************



\newpage
\section*{结论}
\newcounter{结论编号}   %创建一个计数器,这个计数用于给结论章节编号                     
\setcounter{结论编号}{\value{section}} %计数器就像变量一样
\addtocounter{结论编号}{1}
\addcontentsline{toc}{section}{\arabic{结论编号}\quad 结论\tiny{\quad.\quad.\quad.\quad.\quad.\quad.\quad.\quad.\quad.\quad.\quad.\quad.\quad.\quad.\quad.\quad.\quad.\quad.\quad.\quad.\quad.\quad.\quad.\quad.\quad.\quad.\quad.\quad.\quad.\quad.\quad.\quad.\quad.\quad.\quad.\quad.\quad.\quad.\quad.\quad.\quad.\quad.\quad.\quad.\quad}}
*****************************************************************


\newpage
\section*{致谢}
\addcontentsline{toc}{section}{致谢\tiny{\quad.\quad.\quad.\quad.\quad.\quad.\quad.\quad.\quad.\quad.\quad.\quad.\quad.\quad.\quad.\quad.\quad.\quad.\quad.\quad.\quad.\quad.\quad.\quad.\quad.\quad.\quad.\quad.\quad.\quad.\quad.\quad.\quad.\quad.\quad.\quad.\quad.\quad.\quad.\quad.\quad.\quad.\quad.\quad.\quad.\quad.\quad}}

***********************************************************************************************************************************************
\par
\vspace{5ex}
\rightline{\zihao{3}{苏伟强\quad\qquad}}
\rightline{二O一九年五月十九日于广州}
\newpage
}

%\begin{figure}[H]
%	\centering
%	\includegraphics[scale=0.81]{DSPswj.png} %1.png是图片文件的相对路径
%	\caption{DSP上位机程序} %caption是图片的标题
%	\label{DSPswj} %此处的label相当于一个图片的专属标志,目的是方便上下文的引用
%\end{figure}






%参考文献
\renewcommand\refname{参考文献}
\begin{thebibliography}{0}
\addcontentsline{toc}{section}{参考文献\tiny{\quad.\quad.\quad.\quad.\quad.\quad.\quad.\quad.\quad.\quad.\quad.\quad.\quad.\quad.\quad.\quad.\quad.\quad.\quad.\quad.\quad.\quad.\quad.\quad.\quad.\quad.\quad.\quad.\quad.\quad.\quad.\quad.\quad.\quad.\quad.\quad.\quad.\quad.\quad.\quad.\quad.\quad.\quad.\quad}}
%\bibitem{Butterworth}Butterworth S. On the theory of filter amplifiers[J]. Wireless Engineer, 1930, 7(6): 536-541.
%
\bibitem{数字信号处理教材}程佩青. 数字信号处理教程[M]. 清华大学出版社有限公司, 2001.
%
%\bibitem{信号与系统}陈后金. 信号与系统[M]. 清华大学出版社有限公司, 2003.
%
%\bibitem{数字信号处理教材陈}陈后金. 数字信号处理.2版[M]. 高等教育出版社, 2008.
\end{thebibliography}

\end{document}

%%%%%%%%%%%%%%%%%%%%%%%%%%%%%%%%%%%%%%
%%%%%%%%%%%%
%%%%%%%%%%%%双并列图片示例
%%%%%%%%%%%%
%%%%%%%%%%%%%%%%%%%%%%%%%%%%%%%%%%%%%%
%\begin{figure}[H]
%	\centering
%	\begin{minipage}[t]{0,40\textwidth}	
%		\centering
%		\includegraphics[scale=0.5]{ccjg.pdf} %1.png是图片文件的相对路径
%		\caption{IEEE 802.11层次结构} %caption是图片的标题
%		\label{p_ccjg} %此处的label相当于一个图片的专属标志,目的是方便上下文的引用
%	\end{minipage}
%	\hfil
%	\begin{minipage}[t]{0,50\textwidth}	
%		\centering
%		\includegraphics[scale=1]{AODV.pdf} %1.png是图片文件的相对路径
%		\caption{AODV示意图} %caption是图片的标题
%		\label{p_AODV} %此处的label相当于一个图片的专属标志,目的是方便上下文的引用
%	\end{minipage}
%\end{figure}


%%%%%%%%%%%%%%%%%%%%%%%%%%%%%%%%%%%%%%
%%%%%%%%%%%%
%%%%%%%%%%%%表格示例
%%%%%%%%%%%%
%%%%%%%%%%%%%%%%%%%%%%%%%%%%%%%%%%%%%%
%\begin{table}[H]
%	\centering
%	\caption{802.11a/b/g物理层,MAC层参数}
%	\begin{tabular}{ccccc}
%		\toprule
%		&  参数  & 802.11a & 802.11b & 802.11g \\
%		\midrule
%		\multirow{4}[7]{*}{物理层} & 频带/Hz(freq\_) & $5*10^9$ & $2.4*10^9$ & $2.4*10^9$ \\
%		\cmidrule{3-5}       & 通信感知范围\cite{bib13}(CSThresh\_) & $3.17291*10^9$ & $2.79*10^9$ & $2.79*10^9$ \\
%		\cmidrule{3-5}       & 可通信范围\cite{bib13}(RXThresh\_) & $6.5556*10^{10}$ & $5.76*10^9$ & $5.76*10^9$ \\
%		\cmidrule{3-5}       & 传输功率/W(Pt\_) & 0.281838 & 0.281838 & 0.281838 \\
%		\midrule
%		\multirow{9}[17]{*}{MAC层} & 竞争窗口最小值\cite{bib12}/s(CWMin) & 15 & 31 & 15 \\
%		\cmidrule{3-5}       & 竞争窗口最大值\cite{bib12}/s(CWMax) & 1023 & 1023 & 1023 \\
%		\cmidrule{3-5}       & 时隙\cite{bib11}/s(SlotTime\_) & 0.00005 & 0.00002 & 0.000009s \\
%		\cmidrule{3-5}       & SIFS\cite{bib14}\cite{bib11}/s(SIFS\_) & 0.000016 & 0.00001 & 0.000016s \\
%		\cmidrule{3-5}       & 前导码长度\cite{bib14}(PreambleLength) & 96 & 144 & 120 \\
%		\cmidrule{3-5}       & PLCP头部长度\cite{bib14}PLCPHeaderLength\_) & 24 & 48 & 24 \\
%		\cmidrule{3-5}       & PLCP数据率\cite{bib14}/bps(PLCPDataRate\_) & $6*10^6$ & $1*10^6$ & $6*10^6$ \\
%		\cmidrule{3-5}       & 最高速率\cite{bib14}/bps(dataRate) & $5.4*10^7$ & $1.1*10^7$ & $5.4*10^7$ \\
%		\cmidrule{3-5}       & 最低速率\cite{bib14}/bps(basicRate\_) & $6*10^6$ & $1*10^6$ & $6*10^6$ \\
%		\bottomrule
%	\end{tabular}%
%	\label{t_abgcs}%
%\end{table}%


%%%%%%%%%%%%%%%%%%%%%%%%%%%%%%%%%%%%%%
%%%%%%%%%%%%
%%%%%%%%%%%%插入代码示例
%%%%%%%%%%%%title:代码文件标题
%%%%%%%%%%%%language:语言,C++,C,Matlab,Python
%%%%%%%%%%%%%%%%%%%%%%%%%%%%%%%%%%%%%%
%插入代码的时候需要知:注释中同时出现标点符号,英文,中文时会互相影响,
%这个时候,在标点符号,英文后面都要追加空格,才能正常显示
%\lstset{language=C++}
%\begin{lstlisting}[title=AODV100.tr]
%
%\end{lstlisting}


%%%%%%%%%%%%%%%%%%%%%%%%%%%%%%%%%%%%%%
%%%%%%%%%%%%
%%%%%%%%%%%%对齐公式示例
%%%%%%%%%%%%
%%%%%%%%%%%%%%%%%%%%%%%%%%%%%%%%%%%%%%

%\begin{align}
%	\label{kk}
%	k&=\dfrac{3Z_{11}^{'}}{2(1-l^2_2)^{3/2}}\\
%	\label{hh}
%	h&=\frac{1}{\pi}\left[Z_{00}-\frac{k\pi}{2}+k\arcsin(l_2)+kl_2\sqrt{1-l^2_2} \right]\\
%	\label{ll} l&=\frac{1}{2}\left[\sqrt{\frac{5Z_{40}^{'}+3Z^{'}_{20}}{8Z_{20}}}+\sqrt{\frac{5Z_{11}^{'}+Z^{'}_{11}}{6Z_{11}}}\right]\\
%	\label{pp}
%	\phi&=\arctan\left[\frac{Im[Z_{n1}]}{Re[Z_{n1}]}\right]
%\end{align}

%%%%%%%%%%%%%%%%%%%%%%%%%%%%%%%%%%%%%%
%%%%%%%%%%%%
%%%%%%%%%%%%表格示例2
%%%%%%%%%%%%
%%%%%%%%%%%%%%%%%%%%%%%%%%%%%%%%%%%%%%
%\begin{table}[H]
%	\centering
%	\caption{NVIDIA$^{\textregistered}$ Jetson TK1配置一览}
%	\vspace{0.5cm}
%	\begin{tabular}{l}
%		\Xhline{1.2pt}
%		Tegra K1 SOC \\
%		NVIDIA$^{\textregistered}$ Kepler$^{\textregistered}$ GPU、192 个 CUDA 核心 \\
%		NVIDIA$^{\textregistered}$ 4-Plus-1™ 四核 ARM$^{\textregistered}$ Cortex™-A15 CPU \\
%		2 GB x16 内存、64 位宽度 \\
%		16 GB 4.51 eMMC 内存 \\
%		1 个 USB 3.0 端口、A  \\
%		1 个 USB 2.0 端口、Micro AB\\
%		1 个半迷你 PCIE 插槽\\
%		1 个完整尺寸 SD/MMC 连接器\\
%		1 个 RTL8111GS Realtek 千兆位以太网局域网 \\
%		1 个 SATA 数据端口 \\
%		1 个完整尺寸 HDMI 端口 \\
%		1 个 RS232 串行端口 \\
%		SPI 4 兆字节引导闪存\\
%		1 个带 Mic In 和 Line Out 的 ALC5639 Realtek 音频编解码器\\
%		以下信号可通过扩展端口获得:DP/LVDS, Touch SPI 1x4 + 1x1 CSI-2, GPIOs, UART, HSIC, I$^2$C
%		\\
%		\Xhline{1.2pt}
%	\end{tabular}%
%	\label{aaa}%
%\end{table}%

%%%%%%%%%%%%%%%%%%%%%%%%%%%%%%%%%%%%%%
%%%%%%%%%%%%
%%%%%%%%%%%%双并列表格示例
%%%%%%%%%%%%
%%%%%%%%%%%%%%%%%%%%%%%%%%%%%%%%%%%%%%
%\begin{table}[H]\footnotesize
%	\centering
%	
%	\begin{minipage}[t]{0,47\textwidth}		
%		\caption{上位机配置清单}
%		\vspace{0.5cm}
%		\centering
%		\begin{tabular}{cc}
%			\Xhline{1.2pt}
%			运行环境 & ubuntu14 (基于Cortex$^{\textregistered}$-A15芯片) \\
%			编程语言 & C/C++ \\
%			第三方库及组件 & GTK2.0,OpenCV2.4.10 \\
%			开发环境 & Qt Creator 与 make工程管理器  \\
%			编译工具链 & NVIDIA$^{\textregistered}$-ARM$^{\textregistered}$编译工具链 \\
%			程序结构 & 模块化结构 \\
%			\Xhline{1.2pt}
%		\end{tabular}%
%		
%		\label{pzqd}%
%	\end{minipage}
%	\hfil
%	\hfil
%	\begin{minipage}[t]{0,47\textwidth}	
%		\centering
%		\caption{上位机功能清单}
%		\vspace{0.5cm}	
%		\begin{tabular}{cc}
%			\Xhline{1.2pt}
%			编号  & \multicolumn{1}{c}{功能描述} \\
%			\Xhline{1.2pt}
%			1   & \multicolumn{1}{c}{可打开/关闭摄像头} \\
%			2   & 可通过摄像头捕获图片为目标图片 \\
%			3   & 可从文件系统内选择图片并载入为目标图片 \\
%			4   & 可以检测目标图片中圆形轮廓的半径和圆心 \\
%			5   & 可以检测目标图片中平行直线的间距 \\
%			6   & 检测算法的参数可自由调整 \\
%			\Xhline{1.2pt}
%		\end{tabular}%
%		\label{gn}%
%	\end{minipage}
%\end{table}%
